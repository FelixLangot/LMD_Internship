\documentclass{article}
\usepackage[legalpaper, margin=2.5cm]{geometry}
\begin{document}
\begin{list}{}{}
    \item Soient $RH_{dyn}^1$ et $RH_{dyn}^2$ la RH prédite par l'appoche dynamique pour les simulations 1 et 2. \\
    \item Soit $RH_{dyn}^2(w_1)$ la RH prédite par l'appoche dynamique pour la simulation 2 mais avec wtot(z) de la simulation 1. \\
    \item Soient $RH_{stat}^1$ et $RH_{stat}^2$ la RH prédite par l'approche statique pour les simulations 1 et 2. 
\end{list}

Alors
~\\
~\\
\begin{eqnarray*}
    & RH_{dyn}^2-RH_{dyn}^1 &= RH_{stat}^2-RH_{stat}^1 + RH_{dyn}^2-RH_{dyn}^2(w_1) + RH_{dyn}^2(w_1)-RH_{dyn}^1-RH_{stat}^2+RH_{stat}^1 \\ \Leftrightarrow
    &\Delta RH_{dyn} &= \Delta RH_{stat} + RH_{dyn}^2-RH_{dyn}^2(w_1) + RH_{dyn}^2(w_1)-RH_{dyn}^1-\Delta RH_{stat} \\ \Leftrightarrow
    &\Delta RH_{dyn} &= \Delta RH_{stat} + RH_{dyn}^2-RH_{dyn}^2(w_1) + RH_{dyn}^2(w_1)-RH_{dyn}^1-\Delta RH_{stat}
\end{eqnarray*}
~\\
~\\
avec $\Delta RH_{dyn} = RH_{dyn}^2-RH_{dyn}^1, \Delta RH_{stat} = RH_{stat}^2-RH_{stat}^1$

$\bar{ABC}$ vs $\overline{ABC}$

\end{document}