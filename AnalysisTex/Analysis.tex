\documentclass{article}
\usepackage[legalpaper, margin=2.5cm]{geometry}
\usepackage[]{amsmath}
\begin{document}
\begin{list}{}{}
    \item Soient $RH_{dyn}^1$ et $RH_{dyn}^2$ la RH prédite par l'appoche dynamique pour les simulations 1 et 2. 
    \item Soit $RH_{dyn}^2(w_1)$ la RH prédite par l'appoche dynamique pour la simulation 2 mais avec $w_1 = w_{tot}(z)$ de la simulation 1. 
    \item Soient $RH_{stat}^1$ et $RH_{stat}^2$ la RH prédite par l'approche statique pour les simulations 1 et 2. 
\end{list}

Alors
~\\
\begin{align}
    \overline{RH}_{dyn}^2-\overline{RH}_{dyn}^1 &= \overline{RH}_{stat}^2-\overline{RH}_{stat}^1 \\
   &+ \overline{RH}_{dyn}^2-\overline{RH}_{dyn}^2(w_1) \\ 
   &+ \overline{RH}_{dyn}^2(w_1)-\overline{RH}_{dyn}^1-\overline{RH}_{stat}^2+\overline{RH}_{stat}^1 
\end{align}
~\\
Ainsi, la différence de RH prédite est décomposée en 3 termes:\\ 
(1) effet de la cohérence verticale des nuages, \\
(2) effet de wtot(z), \\ 
(3) effet de la cohérence temporelle des nuages.\\ ~\\
\textit{L'idéal serait d'arriver à dire, par exemple (chiffres au pif), que la sécheresse de la simulation cyclone par rapport aux cumulonimbus isolés s'explique à 20\% par les nuages plus cohérents verticalement, à 60\% par les nuages plus cohérents temporellement, et à 20\% par l'environnement qui subside plus vite.} \\

Simulations à comparer:

\begin{list}{}{}
    \item 1. Cumulonimbus vs Cumulonimbus + U 
    \item 2. Cyclone vs Cyclone + U ~~~~~~~~~~~~~~~~~~~~~~~~~~~~~~~~~~~\hbox{
        $\left.\begin{array}{lcl}
            ~ \\
            ~\\
            ~ \\
          \end{array}\right\}\begin{array}{lcl}
           \text{Comparaison avec/sans ascendance}
         \end{array}$}
    \item 3. Squall line vs Squall line + U
    \item ~\\
    \item 4. Cumulonimbus vs Cyclone
    \item 5. Cumulonimbus vs Squall line     ~~~~~~~~~~~~~~~~~~~~~~~~~~~~~~\hbox{
        $\left.\begin{array}{lcl}
            ~ \\
            ~\\
            ~ \\
          \end{array}\right\}\begin{array}{lcl}
           \text{Comparaison entre organisations}
         \end{array}$}
    \item 6. Cyclone vs Squall line
\end{list}
~\\
$RH$ à calculer:
\begin{list}{}{}
    \item 1. $RH_{dyn}^{Cumu+U}(w_{Cumu})$
    \item 2. $RH_{dyn}^{Cyc+U}(w_{Cyc})$
    \item 3. $RH_{dyn}^{Sqll+U}(w_{Sqll})$
    \item 4. $RH_{dyn}^{Cyc}(w_{Cumu})$
    \item 5. $RH_{dyn}^{Sqll}(w_{Cumu})$
    \item 6. $RH_{dyn}^{Sqll}(w_{Cyc})$
\end{list}
~\\
Travail additionel: 
\begin{list}{}{}
    \item 1. Cumu+U vs Cyclone+U
    \item 2. Cumu+U vs LdG+U
\end{list}
~\\
$RH$ à calculer:
\begin{list}{}{}
    \item 1. $RH_{dyn}^{Cyc+U}(w_{Cumu+U})$
    \item 2. $RH_{dyn}^{Sqll+U}(w_{Cumu+U})$
\end{list}

\end{document}